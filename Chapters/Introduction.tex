% Introduction

\chapter{Introduction} % Main chapter title

\label{Chapter1} % For referencing the chapter elsewhere, use \ref{Chapter1} 

\lhead{Chapter 1. \emph{Introduction}} % This is for the header on each page - perhaps a shortened title
	
%----------------------------------------------------------------------------------------
%	MOTIVATION
%----------------------------------------------------------------------------------------

\section{Motivations}

For the last century, radio frequency (RF) has dominated wireless communications.But RF has been a victim of its own success. The number of mobile devices and embedded systems has increased tremendously as technology becomes a primary need in peoples lives. This explosion of mobile devices comes at a cost:the frequency spectrum is a scarce resource. The massive increase in RF communicating devices is leading to a saturation of the available bandwidth, which will  result in a drop of quality-of-service.

To ameliorate the bandwidth saturation problem, the research community has been exploring other wireless technologies. One of the most promising alternatives is visible light communication. Light is electromagnetic radiation just like RF, but the difference lies in frequency. Because of this frequency, the interaction between light and matter is different on a fundamental level which gives light unique properties.

With the advent of Visible Light Communication (VLC), the widespread exploitation of the visible light spectrum is becoming a reality. VLC enables standard
Light Emitting Diodes (LEDs) to transmit data wirelessly, and this is an important step because LEDs are permeating our daily environments at a very fast pace.

After intensive research into energy consumption, in 2009 the European Union and other countries started measures to phase out incandescent light bulbs in
favor of high efficient LEDs. But it is not only residential and commercial lighting that is being replaced with LEDs, a number of other objects such as car lights,
city lights, billboards, smartphone and laptop screens, price tags, toys and home appliances, are also using LEDs to reduce their energy consumption.

A nice VLC example is indoor positioning. At the 2014 Mobile Wolrd Congress (MWC 14'), the i2Cat foundation demonstrate its indoor location positioning system, using VLC and the phone's ambient light sensor.

Other applications, such navigation assistance for visually impaired, propose an other approach using an external peripheral as VLC receiver, connected to the mobile phone through Bluetooth \citep{bluereceiver}

Thus, considering that VLC can potentially transform any LED device into a wireless transmitter; in the near future, there may be a new generation of objects waiting to be networked in a distributed and multi-hop manner.


%----------------------------------------------------------------------------------------
%	OBJECTIVES
%----------------------------------------------------------------------------------------

\section{Objectives}
However, actual research and VLC implementation on mobile phone are highly limited by receiver hardware. In fact, embedded sensor on smarthpone, such as camera are not appropriate for sensing modulated light. External peripheral, adding an other channel reduce considerably the throughput.
That why, the recent concept of modular smartphone, called Phonebloks (2013) and industrialized as Project Ara y Google, could be the solution.

Project Ara is an effort in Google's Advanced Technology Projects (ATAP) organization to create a modular smartphone platform, with the twin aims of delivering a deep customization experience to users and enabling significantly lowered barrier to entry into the mobile hardware ecosystem.

This project aims to take advantage of this new technology developing a Visible Light Communication receiver for the Ara plartform. Major objectives are listed below :
\begin{itemize}
\item Design and develop a module that support visible light communication.
\item Determine the Ara smartphone possibility for future work on wireless link.
\item Experiment a communication between a LED emitter and the modular smartphone.
\item Improve the emitter driver and 
\end{itemize}


%----------------------------------------------------------------------------------------
%	ORGANISATION
%----------------------------------------------------------------------------------------

\section{Report Organization}

Chapter 1 of this report serves to provide an introduction of the basic concepts and techniques and also shows several designs that are required for the implementation of VLC.  \\
Chapter 2 provides the background needed for the VLC designs. \\ 
Chapter 3 provides the literature review of the VLC technology. \\ 
Chapters 4 and 5 provide our proposal description and the experimental setup and implementation of the models. \\
Chapter 6 presents our results and recommendations for improving the designs as well as the conclusion with suggestions for further improvements in the work.
