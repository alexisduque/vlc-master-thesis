% ProjectD

\chapter{Project Description}

\label{ProjectD}

\lhead{Chapter 5. \emph{Project Description}}

%----------------------------------------------------------------------------------------
%	Ara platform discovery
%----------------------------------------------------------------------------------------

\section{Ara platform discovery}

\subsection{Tests possibilities}

\subsection{Issues and weakness}


%----------------------------------------------------------------------------------------
%	Receiver circuit design
%----------------------------------------------------------------------------------------

\section{Receiver circuit design}

\subsection{Needs description}
\subsection{Photodiode}
\subsection{1st AOP : Current to tension converter}
\subsection{2sd: Low Pass filter}
\subsection{Gain}

%----------------------------------------------------------------------------------------
%	Emitter driver
%----------------------------------------------------------------------------------------

\section{Emitter driver}

\subsection{OOK modulation}
\subsection{4B6B Line coding}
\subsection{Algorithm}


%----------------------------------------------------------------------------------------
%	Receiver CAN and Buffer
%----------------------------------------------------------------------------------------

\section{Receiver CAN and Buffer}

\subsection{Digitalization}
\subsection{Decoding}
\subsection{Buffering}
\subsection{Input/output}

%----------------------------------------------------------------------------------------
%	Ara Android App
%----------------------------------------------------------------------------------------

\section{Ara Android App}

In this part, we describe the Android application that we have developed, to support our VLC receiver module.

\subsection{Application structure and operations}
Our application as been developed used Google Android API 18, to be compliant with the operating system version which has been installed on the AP Board.

The application package has been called respecting Java name convention : edu.upc.entel.wng.vlcAraModule and content 3 classes :

\begin{itemize}
\item VLCAraActivity : initialize the application and user interface. It surcharges the Activity class , as requested by the Android API.
\item Sensor : define and perform operations to communicate with our module such as I2C bus initialization, data polling as well as handling and message with the User Interface (UI).
\item VlcLogger : this class realize logging operation saving received data on the board internal storage or an external SD Card.
\end{itemize}

Application workflow is quiet simple :
\begin{enumerate}
\item Initialize and start the application main Activity.
\item Setup the I2C Bus.
\item Initialize the logger and create a log file.
\item Start the \textbf{Sensor} thread : poll I2C bus at defined intervall 
\item Handle Sensor Thread message and user interaction
\end{enumerate}

\subsection{User Interface}

The application user interface as few components as defined in the XML activity layout definition file :
\begin{itemize}
\item EditText : used to display received bits.
\item 
\end{itemize}

\subsection{I2C JNI interface}
\subsection{Results}
\subsection{Interpretation}