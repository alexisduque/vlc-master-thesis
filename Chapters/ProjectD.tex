% ProjectD

\chapter{Project Description}

\label{ProjectD}

\lhead{Chapter 5. \emph{Project Description}}

%----------------------------------------------------------------------------------------
%	Ara platform discovery
%----------------------------------------------------------------------------------------

\section{Ara platform study}

In this section, we will describe different studies and preliminary experimentation we've done before designing  the module.

In fact, because of the lake of literature or past experience with the Ara platform, we first need to study the Ara Module Development Kit send by Google ATAP. We focus on understanding the role of each board, and there I/O - regarding protocols and speed. Then we determine the reliable rate on witch the operating system can execute a single task - such as I/O operation. Finally we perform some benchmark and CPU load test. 

\subsection{MDK and boards study}



\subsection{Android for ARA reliable I/O operation rate}

\subsection{Ara benchmark}

\subsection{Conclusion}


%----------------------------------------------------------------------------------------
%	Receiver circuit design
%----------------------------------------------------------------------------------------

\section{Receiver circuit design}

\subsection{Needs description}
\subsection{Photodiode}
\subsection{1st AOP : Current to tension converter}
\subsection{2sd: Low Pass filter}
\subsection{Gain}

%----------------------------------------------------------------------------------------
%	Emitter driver
%----------------------------------------------------------------------------------------

\section{Emitter driver}

\subsection{OOK modulation}
\subsection{4B6B Line coding}
\subsection{Algorithm}


%----------------------------------------------------------------------------------------
%	Receiver CAN and Buffer
%----------------------------------------------------------------------------------------

\section{Receiver CAN and Buffer}

\subsection{Digitalization}
\subsection{Decoding}
\subsection{Buffering}
\subsection{Input/output}

%----------------------------------------------------------------------------------------
%	Ara Android App
%----------------------------------------------------------------------------------------

\section{Ara Android App}

In this part, we describe the Android application that we have developed, to support our VLC receiver module.

\subsection{Application structure and operations}
Our application as been developed used Google Android API 18, to be compliant with the operating system version which has been installed on the AP Board.

The application package has been called respecting Java name convention : edu.upc.entel.wng.vlcAraModule and content 3 classes :

\begin{itemize}
\item VLCAraActivity : initialize the application and user interface. It surcharges the Activity class , as requested by the Android API.
\item Sensor : define and perform operations to communicate with our module such as I2C bus initialization, data polling as well as handling and message with the User Interface (UI).
\item VlcLogger : this class realize logging operation saving received data on the board internal storage or an external SD Card.
\end{itemize}

Application workflow is quiet simple :
\begin{enumerate}
\item Initialize and start the application main Activity.
\item Setup the I2C Bus.
\item Initialize the logger and create a log file.
\item Start the \textbf{Sensor} thread that poll I2C bus at defined interval and send the result to the android activity. 
\item Handle Sensor thread messaging and user interaction.
\end{enumerate}

\subsection{User Interface}

The application user interface as few components as defined in the XML activity layout description file :
\begin{itemize}
\item EditText field: used to display received bits.
\item "Clear" and "Save" Button :
\end{itemize}

Each component are placed in a horizontal layout container.

\subsection{I2C JNI interface}

As the Android Java API for the Ara Module Development Kit is the same as the standard API, we need to implement additional components in order to access the low level hardware such as the GPIO or the I2C bus.
The Ara MDK is provided with an operating system level driver, developed in C, that can be used to configure, read and write date, in a simple way.
However, this C interface, can't be directly used through the Java API. In order to give the possibility to perform I2C operations in our Android application, we develop a JNI interface that would wrap the C driver into a Java package.
So we define 2 Java classes : 
\begin{itemize}
\item I2CManager : it configures the i2c bus, by using UNIX I2C kernel driver and execute I2CTransaction.
\item I2CTransaction : it represents read or write operation on the bus.
\end{itemize}

Class methods are detailed in the Anexxe.

\subsection{Results}
\subsection{Interpretation}