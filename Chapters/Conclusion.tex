% Conclusion

\chapter{Conclusion}

\label{Conclusion}

\lhead{Chapter 7. \emph{Conclusion}}

%----------------------------------------------------------------------------------------
%	Achieved work
%----------------------------------------------------------------------------------------

\section{Achieved work}

In this project, we demonstrate the new opportunity and usage that the recent concept of modular smartphone and ARA platform give to the smartphone, extending its hardware to support visible light communication.

We finally develop a VLC receiver module for the ARA smartphone using the development kit provided by Google, with the full communication chain - including the emitter LED driver - in order to test it.

In this way, we were able to give our contribution to the VLC research, probing a modified version of 4B6B runlength-limited coding associated with the On-Off Keying modulation.

Bringing a suitable and dedicated receiver to the smartphone, we overcome past results in term of bitrate, in an indoor transmission scenario, between a lightening LED and a smartphone.


%----------------------------------------------------------------------------------------
%	Future works
%----------------------------------------------------------------------------------------

\section{Future works}

On the one hand, regarding our work and results, we might improve them in different ways. First the circuit need a better expertise to increase it's stability, or interferences tolerance, and obtain a better SNR ratio.
Then, the release of the MDK v0.2 and the new Spiral 2 development board by Google ATAP and the ARA project team, make our work on this board quiet obsolete as it's totally different from the commercialized Ara smartphone.

On other hand, as we didn't develops any protocol at the Medium-Access-Control (MAC) or application layer for our VLC system, a lot of work keep underachieved and need to be investigate in further research projects.

Finally, this proof of concept of a novel wireless communication module for the ARA smartphone, open the door to further modules development to bring  M2M and IoT protocols such 6LoWPAN or Zigbee to the mobile phones.